\chapter{Mathematical preliminaries}

\section{Even and odd functions}
\begin{definition}
    The function $ f : \mathbb{R} \rightarrow \mathbb{R} $ is an even function iff
    \begin{equation}
        f(-x) = f(x), \quad \forall x \in \mathbb{R}.
        \label{eq:definition:even}
    \end{equation}
\end{definition}
The graph of any even function is symmetric with respect to the $ y $-axis.

\begin{lemma}
    Let $ f(x) $ be an integrable even function. Then,
    \begin{equation}
        \int_{-a}^{0} f(x) dx = \int_{0}^{a} f(x) dx, \quad \forall a \in \mathbb{R},
    \end{equation}
    and therefore
    \begin{equation}
        \int_{-a}^{a} f(x) dx = 2 \int_{0}^{a} f(x) dx, \quad \forall a \in \mathbb{R}.
    \end{equation}
    Moreover, if $ \int_{0}^{\infty} f(x) dx $ exists, then
    \begin{equation}
        \int_{-\infty}^{0} f(x) dx = \int_{0}^{\infty} f(x) dx,
    \end{equation}
    and
    \begin{equation}
        \int_{-\infty}^{\infty} f(x) dx = 2 \int_{0}^{\infty} f(x) dx.
    \end{equation}
\end{lemma}

\begin{definition}
    The function $ f : \mathbb{R} \rightarrow \mathbb{R} $ is an odd function iff
    \begin{equation}
        f(-x) = -f(x), \quad \forall x \in \mathbb{R}.
        \label{eq:definition:odd}
    \end{equation}
\end{definition}
If we let $ x = 0 $ in \eqref{eq:definition:odd}, we find that $ f(0) = 0 $ for any odd function $ f(x) $.
Also, the graph of any odd function is symmetric with respect to the point (0, 0).

\begin{lemma}
    Let $ f(x) $ be an integrable odd function. Then,
    \begin{equation}
        \int_{-a}^{a} f(x) dx = 0, \quad \forall a \in \mathbb{R}.
    \end{equation}
    Moreover, if $ \int_{0}^{\infty} f(x) dx $ exists, then
    \begin{equation}
        \int_{-\infty}^{\infty} f(x) dx = 0.
    \end{equation}
\end{lemma}

\section{Useful sums with interesting proofs}
The following sums occur frequently when estimating operation counts of numerical algorithms:
\begin{align}
    \sum_{k=1}^{n} k &= \frac{n (n + 1)}{2}; \label{eq:sum-pow1}\\
    \sum_{k=1}^{n} k^2 &= \frac{n (n + 1) (2 n + 1)}{6}; \label{eq:sum-pow2}\\
    \sum_{k=1}^{n} k^3 &= \left( \frac{n (n + 1)}{2} \right)^2. \label{eq:sum-pow3}
\end{align}

\section{Sequences satisfying linear recursions}
\begin{definition}
    A sequence $ (x_n)_{n \geq 0} $ satisfies a linear recursion of order $ k $ iff there exist constants $ a_i $, $ i = 0 : k $ with $ a_k \neq 0 $, such that
    \begin{equation}
        \sum_{i=0}^{k} a_i x_{n+i} = 0, \quad \forall n \geq 0.
        \label{eq:definition:linear-recursion}
    \end{equation}
\end{definition}

The recursion \eqref{eq:definition:linear-recursion} is called a linear recursion because of the following linearity properties:\\
(i) If the sequence $ (x_n)_{n \geq 0} $ satisfies the linear recursion \eqref{eq:definition:linear-recursion}, then the sequence $ (z_n)_{n \geq 0} $ given by
\begin{equation}
    z_n = C x_n, \quad \forall n \geq 0,
\end{equation}
where $ C $ is an arbitrary constant, also satisfies the linear recursion \eqref{eq:definition:linear-recursion}.\\
(ii) If the sequences $ (x_n)_{n \geq 0} $ and $ (y_n)_{n \geq 0} $ satisfies the linear recursion \eqref{eq:definition:linear-recursion}, then the sequence $ (z_n)_{n \geq 0} $ given by
\begin{equation}
    z_n = x_n + y_n, \quad \forall n \geq 0,
\end{equation}
also satisfies the linear recursion \eqref{eq:definition:linear-recursion}.

\begin{definition}
    The characteristic polynomial $ P(z) $ corresponding to the linear recursion $ \sum_{i=0}^{k} a_i x_{n+i} = 0 $, for all $ n \geq 0 $, is defined as
    \begin{equation}
        P(z) = \sum_{i=0}^{k} a_i z^i.
    \end{equation}
\end{definition}

$ P(z) $ is a polynomial of degree $ k $, i.e., $ \deg(P(z)) = k $.
If $ P(z) $ has $ p $ different roots, $ \lambda_j $, $ j = 1 : p $, with $ p \leq k $, and if $ m(\lambda_j) $ denotes the multiplicity of the root $ \lambda_j $, then $ \sum_{j=1}^{p} m(\lambda_j) = k $ where $ \lambda_j $ can be a complex number.

\begin{theorem}\label{thm:linear-recursion}
    Let $ (x_n)_{x \geq 0} $ be a sequence satisfying the linear recursion
    \begin{equation}
        \sum_{i=0}^{k} a_i x_{n+i} = 0, \quad \forall n \geq 0,
        \label{eq:theorem:linear-recursion}
    \end{equation}
    with $ a_k \neq 0 $, and let $ P(z) = \sum_{i=0}^{k-1} a_i z^i $ be the characteristic polynomial associated with recursion \eqref{eq:theorem:linear-recursion}.
    Let $ \lambda_j $, $ j = 1 : p $, where $ p \leq k $, be the roots of $ P(z) $, and let $ m(\lambda_j) $ be the multiplicity of $ \lambda_j $.
    The general form of the sequence $ (x_n)_{n \geq 0} $ satisfying the linear
    recursion \eqref{eq:theorem:linear-recursion} is
    \begin{equation}
        x_n = \sum_{j=1}^{p} \left( \sum_{i=0}^{m(\lambda_j)-1} C_{i,j} n^i \right) \lambda_j^n, \quad \forall n \geq 0,
        \label{eq:theorem:linear-recursion-sequence}
    \end{equation}
    where $ C_{i,j} $ are constant numbers.
\end{theorem}

\section{The ``Big O" and ``little o" notations}
\begin{definition}
    Let $ f,g : \mathbb{R} \rightarrow \mathbb{R} $.
    We write that $ f(x) = O(g(x) $, as $ x \rightarrow \infty $, iff there exist constants $ C > 0 $ and $ M > 0 $ such that $ \left| \frac{f(x)}{g(x)} \leq C \right| $, for any $ x \geq M $.
    This can be written equivalently as
    \begin{equation}
        f(x) = O(g(x)), \quad \text{as} \quad x \rightarrow \infty, \quad \text{iff} \quad \limsup_{x \rightarrow \infty} \left| \frac{f(x)}{g(x)} \right| < \infty.
        \label{eq:definition:big-o}
    \end{equation}
\end{definition}

The ``little o" notation refers to functions whose ratios tend to 0 at certain fpoints, and can be defined for $ x \rightarrow \infty $, $ x \rightarrow a $, and $ x \rightarrow -\infty $ as follows:
\begin{definition}
    Let $ f,g : \mathbb{R} \rightarrow \mathbb{R} $.
    Then
    \begin{align}
        f(x) &= o(g(x)), \quad \text{as} \quad x \rightarrow \infty, \quad \text{iff} \quad \lim_{x \rightarrow \infty} \left| \frac{f(x)}{g(x)} \right| = 0; \\
        f(x) &= o(g(x)), \quad \text{as} \quad x \rightarrow -\infty, \quad \text{iff} \quad \lim_{x \rightarrow -\infty} \left| \frac{f(x)}{g(x)} \right| = 0; \\
        f(x) &= o(g(x)), \quad \text{as} \quad x \rightarrow a, \quad \text{iff} \quad \lim_{x \rightarrow a} \left| \frac{f(x)}{g(x)} \right| = 0.
    \end{align}
\end{definition}


\section{Exercises}
\subsection{Question 1}
Let $ f : \mathbb{R} \rightarrow \mathbb{R} $ be an odd function.

\subsubsection{Part (i)}
Show that $ x f(x) $ is an even function and $ x^2 f(x) $ is an odd function.

\paragraph{Answer}
Let $ g(x) = x f(x) $, using definition \eqref{eq:definition:odd}, we have
\begin{align*}
    g(-x) &= -x f(-x) \\
          &= -x (-f(x)) \\
          &= x f(x) \\
          &= g(x).
\end{align*}
Therefore, $ x f(x) $ is an even function.

Let $ g(x) = x^2 f(x) $, using definition \eqref{eq:definition:odd}, we have
\begin{align*}
    g(-x) &= (-x)^2 f(-x) \\
          &= x^2 (-f(x)) \\
          &= -x^2 f(x) \\
          &= -g(x)
\end{align*}
Therefore, $ x^2 f(x) $ is an odd function.

\subsubsection{Part (ii)}
Show that the function $ g_1 : \mathbb{R} \rightarrow \mathbb{R} $ given by $ g_1(x) = f(x^2) $ is an even function and that the function $ g_2 : \mathbb{R} \rightarrow \mathbb{R} $ given by $ g_2(x) = f(x^3) $ is an odd function.

\paragraph{Answer}
\begin{align*}
    g_1(-x) &= f((-x)^2) \\
            &= f(x^2) \\
            &= g_1(x)
\end{align*}
Therefore $ g_1 $ is an even function.

Let $ y = x^3 $:
\begin{align*}
    g_2(-x) &= f((-x)^3) \\
            &= f(-x^3) \\
            &= f(-y) \\
            &= -f(y) \\
            &= -f(x^3) \\
            &= -g_2(x)
\end{align*}
Therefore $ g_2 $ is an odd function.

\subsubsection{Part (iii)}
Let $ i $ be even, $ j $ be odd, and $ y = x^j $:
\begin{align*}
    h(-x) &= (-x)^i f((-x)^j) \\
          &= x^i f(-x^j) \\
          &= x^i f(-y) \\
          &= -x^i f(y) \\
          &= -x^i f(x^j) \\
          &= -h(x)
\end{align*}

Let $ i $ be odd, $ j $ be even:
\begin{align*}
    h(-x) &= (-x)^i f((-x)^j) \\
          &= -x^i f(x^j) \\
          &= -h(x)
\end{align*}

When $ i + j $ is odd, $ h(x) $ is an odd function.

\subsection{Question 2}
Let $ S(n,2) = \sum_{k=1}^{n} k^2 $ and $ S(n,3) = \sum_{k=1}^{n} k^3 $.

\subsubsection{Part (i)}
Let $ T(n,2,x) = \sum_{k=1}^{n} k^2 x^k $. Use formulas,
\begin{equation*}
    T(n, 2, x) = x \frac{d}{dx} (T(n, 1, x)),
\end{equation*}
and
\begin{equation*}
    T(n, 1, x) = \frac{x - (n + 1) x^{n + 1} + n x^{n + 2}}{(1 - x)^2},
\end{equation*}
to show that
\begin{equation*}
    T(n, 2, x) = \frac{x + x^2 - (n + 1)^2 x^{n + 1} + (2 n^2 + 2 n - 1) x^{n + 2} - n^2 x^{n + 3}}{(1 - x)^3}.
\end{equation*}

\paragraph{Answer}
Using quotient rule,
\begin{align*}
    T(n, 2, x) &= x \frac{d}{dx} (T(n, 1, x)) \\
               &= x \frac{d}{dx} \left( \frac{x - (n + 1) x^{n + 1} + n x^{n + 2}}{(1 - x)^2} \right) \\
               &= x \left( \frac{\frac{d}{dx} (x - (n + 1) x^{n + 1} + n x^{n + 2}) (1 - x)^2 - (x - (n + 1) x^{n + 1} + n x^{n + 2}) \frac{d}{dx} (1 - x)^2}{(1 - x)^4} \right) \\
               &= x \left( \frac{(1 - (n + 1)^2 x^n + (n^2 + 2 n) x^{n + 1})(1 - x)^2 - (x - (n + 1) x^{n + 1} + n x^{n + 2}) (-2(1 - x))}{(1 - x)^4} \right) \\
               &= x \left( \frac{(1 - (n + 1)^2 x^n + (n^2 + 2 n) x^{n + 1})(1 - x)^2 + 2 (x - (n + 1) x^{n + 1} + n x^{n + 2}) (1 - x)}{(1 - x)^4} \right) \\
               &= x \left( \frac{(1 - (n + 1)^2 x^n + (n^2 + 2 n) x^{n + 1})(1 - x) + 2 (x - (n + 1) x^{n + 1} + n x^{n + 2})}{(1 - x)^3} \right) \\
               &= x \left( \frac{(1 - (n + 1)^2 x^n + (n^2 + 2 n) x^{n + 1}) - (x - (n + 1)^2 x^{n + 1} + (n^2 + 2 n) x^{n + 2}) + \cdots}{(1 - x)^3} \right) \\
               &= x \left( \frac{1 - x - (n + 1)^2 x^n + (2n^2 + 4 n + 1) x^{n + 1} - (n^2 + 2 n) x^{n + 2} + \cdots}{(1 - x)^3} \right) \\
               &= x \left( \frac{1 + x - (n + 1)^2 x^n + (2n^2 + 2 n - 1) x^{n + 1} - n^2 x^{n + 2}}{(1 - x)^3} \right) \\
               &= \frac{x + x^2 - (n + 1)^2 x^{n + 1} + (2n^2 + 2 n - 1) x^{n + 2} - n^2 x^{n + 3}}{(1 - x)^3}
\end{align*}

\subsubsection{Part (ii)}
Note that $ S(n,2) = T(n,2,1) $.
Use l'H\^{o}pitals's rule to evaluate $ T(n,2,1) $, and conclude that $ S(n,2) = \frac{n(n+1)(2n+1)}{6} $.

\paragraph{Answer}
\begin{align*}
    \lim_{x \rightarrow 1} T(n, 2, x) &= \lim_{x \rightarrow 1} \frac{x + x^2 - (n + 1)^2 x^{n + 1} + (2n^2 + 2 n - 1) x^{n + 2} - n^2 x^{n + 3}}{(1 - x)^3} \\
                                      &= \lim_{x \rightarrow 1} \frac{(n - 1) n (n + 1)^3 x^{n - 2} - n(n + 1)(n + 2)(2 n^2 + 2 n - 1) x^{n - 1} + (n + 1)(n + 2)(n + 3) n^2 x^n}{6} \\
                                      &= \frac{(n - 1) n (n + 1)^3 - n(n + 1)(n + 2)(2 n^2 + 2 n - 1) + (n + 1)(n + 2)(n + 3) n^2}{6} \\
                                      &= \frac{n (n + 1) (2 n + 1)}{6} \\
    S(n, 2) &= \frac{n (n + 1) (2 n + 1)}{6} \\
\end{align*}

\subsubsection{Part (iii)}
Compute $ T(n,3,x) = \sum_{k=1}^{n} k^3 x^k $ using the formula
\begin{equation*}
    T(n, 3, x) = x \frac{d}{dx} (T(n, 2, x)).
\end{equation*}

\paragraph{Answer}
\begin{align*}
    T(n, 3, x) &= x \frac{d}{dx} (T(n, 2, x)) \\
               &= x \frac{d}{dx} \frac{x + x^2 - (n + 1)^2 x^{n + 1} + (2 n^2 + 2 n - 1) x^{n + 2} - n^2 x^{n + 3}}{(1 - x)^3}\\
               &= \frac{x((n^3 (x - 1)^3 - 3 n^2 (x - 1)^2 + 3 n (x^2 - 1) - x (x + 4) - 1) x^n + (x + 4) x + 1)}{(1 - x)^4}
\end{align*}

\subsubsection{Part (iv)}
Note that $ S(n,3) = T(n,3,1) $.
Use l'H\^{o}pital's rule to evaluate $ T(n,3,1) $, and conclude that $ S(n,3) = \left( \frac{n(n+1)}{2} \right)^2 $.
\begin{align*}
    \lim_{x \rightarrow 1} T(n, 3, x) &= \lim_{x \rightarrow 1} \frac{x((n^3 (x - 1)^3 - 3 n^2 (x - 1)^2 + 3 n (x^2 - 1) - x (x + 4) - 1) x^n + (x + 4) x + 1)}{(1 - x)^4} \\
\end{align*}

\subsection{Question 3}
Compute $ S(n, 4) = \sum_{k=1}^{n} k^4 $ using the recursion formula for $ i = 4 $, the fact that $ S(n, 0) = n $, and formulas for $ S(n, 1) $, $ S(n, 2) $, and $ S(n, 3) $.

\paragraph{Answer}
\begin{align*}
    S(n, 4) &= \frac{1}{5} \left( (n + 1)^5 - 1 - \sum_{j=0}^{3} \binom{5}{j} S(n, j) \right) \\
            &= \frac{1}{5} \left( (n + 1)^5 - 1 - S(n, 0) - 5 S(n, 1) - 10 S(n, 2) - 10 S(n, 3) \right) \\
            &= \frac{1}{5} \left( (n + 1)^5 - 1 - n - 5 \frac{n (n + 1)}{2} - 10 \frac{n (n + 1)(2 n + 1)}{6} - 10 \left( \frac{n (n + 1)}{2} \right)^2 \right) \\
            &= \frac{n (n + 1)(2 n + 1)(3 n^2 + 3 n - 1)}{30}
\end{align*}

\subsection{Question 4}
It is easy to see that the sequence $ (x_n)_{n \geq 1} $ given by $ x_n = \sum_{k=1}^{n} k^2 $ satisfies the recursion
\begin{equation}
    x_{n + 1} = x_n + (n + 1)^2, \quad \forall n \geq 1,
    \label{eq:exercise:ch0-q4}
\end{equation}
with $ x_1 = 1 $.

\subsubsection{Part (i)}
By substituting $ n + 1 $ for $ n $ in \eqref{eq:exercise:ch0-q4}, obtain that
\begin{equation}
    x_{n + 2} = x_{n + 1} + (n + 2)^2.
    \label{eq:exercise:ch0-q4i-1}
\end{equation}
Substract \eqref{eq:exercise:ch0-q4} from \eqref{eq:exercise:ch0-q4i-1} to find that
\begin{equation}
    x_{n + 2} = 2 x_{n + 1} - x_n + 2 n + 3, \quad \forall n \geq 1,
    \label{eq:exercise:ch0-q4i-2}
\end{equation}
with $ x_1 = 1 $ and $ x_2 = 5 $.

\paragraph{Answer}
\begin{align*}
    x_{(n + 1) + 1} &= x_{n + 1} ((n + 1) + 1)^2 \\
    x_{n + 2} &= x_{n + 1} (n + 2)^2
\end{align*}
Substract \eqref{eq:exercise:ch0-q4} from \eqref{eq:exercise:ch0-q4i-1}:
\begin{align*}
    x_{n + 2} - x_{n + 1} &= x_{n + 1} + (n + 2)^2 - x_n - (n + 1)^2 \\
    x_{n + 2} &= 2 x_{n + 1} - x_n + n^2 + 4 n + 4 - n^2 - 2 n - 1 \\
              &= 2 x_{n + 1} - x_n + 2 n + 3
\end{align*}

\subsubsection{Part (ii)}
Similarly, substitute $ n + 1 $ for $ n $ in \eqref{eq:exercise:ch0-q4i-2} and obtain that
\begin{equation}
    x_{n + 3} = 2 x_{n + 2} - x_{n + 1} + 2(n + 1) + 3.
    \label{eq:exercise:ch0-q4ii-1}
\end{equation}
Substract \eqref{eq:exercise:ch0-q4i-2} from \eqref{eq:exercise:ch0-q4ii-1} to find that
\begin{equation}
    x_{n + 3} = 3 x_{n + 2} - 3 x_{n + 1} + x_n + 2, \quad \forall n \geq 1,
    \label{eq:exercise:ch0-q4ii-2}
\end{equation}
with $ x_1 = 1 $, $ x_2 = 5 $, and $ x_3 = 14 $.

\paragraph{Answer}
\begin{align*}
    x_{(n + 1) + 2} &= 2 x_{(n + 1) + 1} - x_{n + 1} + 2(n + 1) + 3 \\
    x_{n + 3} &= 2 x_{n + 2} - x_{n + 1} + 2(n + 1) + 3
\end{align*}
Substract \eqref{eq:exercise:ch0-q4i-2} from \eqref{eq:exercise:ch0-q4ii-1}
\begin{align*}
    x_{n + 3} - x_{n + 2} &= 2 x_{n + 2} - x_{n + 1} + 2(n + 1) + 3 - 2 x_{n + 1} + x_n - 2 n - 3 \\
    x_{n + 3} &= 3 x_{n + 2} -3 x_{n + 1} + x_n + 2
\end{align*}

\subsubsection{Part (iii)}
Use a similar method to prove that the sequence $ (x_n)_{n \geq 0} $ satisfies the linear recursion
\begin{equation}
    x_{n + 4} - 4 x_{n + 3} + 6 x_{n + 2} - 4 x_{n + 1} + x_n = 0, \quad \forall n \geq 1.
    \label{eq:exercise:ch0-q4iii}
\end{equation}
The characteristic polynomial associated to the recursion \eqref{eq:exercise:ch0-q4iii} is
\begin{equation*}
    P(z) = z^4 - 4 z^3 + 6 z^2 - 4 z + 1 = (z - 1)^4.
\end{equation*}
Use the fact that $ x_1 = 1 $, $ x_2 = 5 $, $ x_3 = 14 $, and $ x_4 = 30 $ to show that
\begin{equation*}
    x_n = \frac{n (n + 1) (2 n + 1)}{6}, \quad \forall n \geq 1,
\end{equation*}
and conclude that
\begin{equation*}
    S(n, 2) = \sum_{k=1}^{n} k^2 = \frac{n (n + 1) (2 n + 1)}{6}, \quad \forall n \geq 1.
\end{equation*}

\paragraph{Answer}
Substitute $ n + 1 $ for $ n $ in \eqref{eq:exercise:ch0-q4ii-2} to obtain
\begin{equation}
    x_{n + 4} = 3 x_{n + 3} - 3 x_{n + 2} + x_{n + 1} + 2.
    \label{eq:exercise:ch0-q4iii-ans-1}
\end{equation}
Substract \eqref{eq:exercise:ch0-q4ii-2} from \eqref{eq:exercise:ch0-q4iii-ans-1} to obtain that
\begin{align*}
    x_{n + 4} - x_{n + 3} &= 3 x_{n + 3} - 3 x_{n + 2} + x_{n + 1} + 2 - (3 x_{n + 2} - 3 x_{n + 1} + x_n + 2) \\
    x_{n + 4} - 4 x_{n + 3} + 6 x_{n + 2} - 4 x_{n + 1} - x_n &= 0
\end{align*}

The characteristic polynomial has root $ \lambda = 1 $ with multiplicity 4.
The linear recurssion can be expressed as
\begin{align*}
    x_n &= \sum_{j=1}^{p} \left( \sum_{i=0}^{3} C_{i,j} n^i \right) \lambda_j^n \\
        &= \sum_{j=1}^{p} \left( C_{0,j} + C_{1,j} n + C_{2,j} n^2 + C_{3,j} n^3 \right) \lambda_j^n \\
        &= C_1 + C_2 n + C_3 n^2 + C_4 n^3
\end{align*}
Since $ x_1 = 1 $, $ x_2 = 5 $, $ x_3 = 14 $, and $ x_4 = 30 $, $ C_1 $, $ C_2 $, $ C_3 $, and $ C_4 $ must solve the linear system
\begin{equation*}
    \begin{pmatrix}
        1 & 1 & 1 & 1 \\
        1 & 2 & 4 & 8 \\
        1 & 3 & 9 & 27 \\
        1 & 4 & 16 & 64
    \end{pmatrix}
    \begin{pmatrix}
        C_1 \\
        C_2 \\
        C_3 \\
        C_4
    \end{pmatrix}
    =
    \begin{pmatrix}
        1 \\
        5 \\
        14 \\
        30
    \end{pmatrix}.
\end{equation*}
We obtain that $ C_1 = 0 $, $ C_2 = \frac{1}{6} $, $ C_3 = \frac{1}{2} $, and $ C_4 = \frac{1}{3} $ and therefore
\begin{equation*}
    x_n = \frac{n}{6} + \frac{n^2}{2} + \frac{n^3}{3} = \frac{n (n + 1) (n + 2)}{6}
\end{equation*}

\subsection{Question 5}
Find the general form of the sequence $ (x_n)_{n \geq 0} $ satisfying the linear recursion
\begin{equation*}
    x_{n + 3} = 2 x_{n + 1} + x_n, \quad \forall n \geq 0,
\end{equation*}
with $ x_0 = 1 $, $ x_1 = -1 $, and $ x_2 = 1 $.

\paragraph{Answer}
Rewrite the recursion in the form \eqref{eq:definition:linear-recursion} as
\begin{equation*}
    x_{n + 3} - 2 x_{n + 1} - x_n = 0, \quad \forall n \geq 0.
\end{equation*}
The characteristic polynomial associated to the linear recursion is
\begin{align*}
    P(z) &= z^3 - 2z - 1 \\
         &= (z + 1)(z^2 - z - 1)
\end{align*}
and the roots of $ P(z) $ are
\begin{equation*}
    \lambda_1 = -1, \quad \lambda_2 = \frac{1 + \sqrt{5}}{2}, \quad \lambda_3 = \frac{1 - \sqrt{5}}{2}.
\end{equation*}

From Theorem \ref{thm:linear-recursion}, we find that
\begin{equation*}
    x_n = C_1 \lambda_1^n + C_2 \lambda_2^n + C_3 \lambda_3^n, \quad \forall n \geq 0.
\end{equation*}
Given $ x_0 = 1 $, $ x_1 = -1 $, and $ x_2 = 1 $, we obtain the linear system
\begin{equation*}
    \begin{pmatrix}
        1 & 1 & 1 \\
        \lambda_1 & \lambda_2 & \lambda_3 \\
        \lambda_1^2 & \lambda_2^2 & \lambda_3^2
    \end{pmatrix}
    \begin{pmatrix}
        C_1 \\
        C_2 \\
        C_3
    \end{pmatrix}
    =
    \begin{pmatrix}
        1 \\
        -1 \\
        1
    \end{pmatrix}
\end{equation*}
By solving the linear system, we find that $ C_1 = 1 $, $ C_2 = 0 $, and $ C_3 = 0 $.
The general formula for is
\begin{equation*}
    x_n = (-1)^n, \quad \forall n \geq 0.
\end{equation*}

\subsection{Question 6}
The sequence $ (x_n)_{n \geq 0} $ satisfies the recursion
\begin{equation*}
    x_{n + 1} = 3 x_n + 2, \quad \forall n \geq 0,
\end{equation*}
with $ x_0 = 1 $.

\subsubsection{Part (i)}
Show that the sequence $ (x_n)_{n \geq 0} $ satisfies the linear recursion
\begin{equation*}
    x_{n + 2} = 4 x_{n + 1} - 3 x_n, \quad \forall n \geq 0,
\end{equation*}
with $ x_0 = 1 $ and $ x_1 = 5 $.

\paragraph{Answer}
Substitute $ n + 1 $ for $ n $ to obtain
\begin{equation*}
    x_{n + 2} = 3 x_{n + 1} + 2
\end{equation*}
Subtract the original recursion to get
\begin{align*}
    x_{n + 2} - x_{n + 1} &= 3 x_{n + 1} + 2 - 3 x_n - 2 \\
    x_{n + 2} &= 4 x_{n + 1} - 3 x_n
\end{align*}

\subsubsection{Part (ii)}
Find the general formula for $ x_n $, $ n \geq 0 $.

\paragraph{Answer}
The characteristic polynomial has the form
\begin{equation*}
    P(z) = z^2 - 4 z + 3 = (z - 1)(z - 3)
\end{equation*}
which has roots $ \lambda_1 = 1 $ and $ \lambda_2 = 3 $.
We obtain the linear system
\begin{equation*}
    \begin{cases}
        C_1 + C_2 = 1; \\
        C_1 \lambda_1 + C_2 \lambda_2 = 5.
    \end{cases}
\end{equation*}
The solution to the linear system is $ C_1 = -1 $ and $ C_2 = 2 $.
Therefore, the general form is
\begin{equation*}
    x_n = 2(3)^n - 1
\end{equation*}

\subsection{Question 7}
The sequence $ (x_n)_{n \geq 0} $ satisfies the recursion
\begin{equation*}
    x_{n + 1} = 3 x_n + n + 2, \quad \forall n \geq 0,
\end{equation*}
with $ x_0 = 1 $.

\subsubsection{Part (i)}
Show that the sequence $ (x_n)_{n \geq 0} $ satisfies the linear recursion
\begin{equation*}
    x_{n + 3} = 5 x_{n + 2} - 7 x_{n + 1} + 3 x_n, \quad \forall n \geq 0,
\end{equation*}
with $ x_0 = 1 $, $ x_1 = 5 $, and $ x_2 = 18 $.

\paragraph{Answer}
Substitute $ n + 1 $ for $ n $, we obtain
\begin{equation*}
    x_{n + 2} = 3 x_{n + 1} + n + 3
\end{equation*}
Subtract
\begin{equation*}
    x_{n + 2} = 4 x_{n + 1} - 3 x_n + 1
\end{equation*}

Substitute $ n + 1 $ for $ n $, we obtain
\begin{equation*}
    x_{n + 3} = 4 x_{n + 2} - 3 x_{n + 1} + 1
\end{equation*}
Subtract
\begin{equation*}
    x_{n + 3} = 5 x_{n + 2} - 7 x_{n + 1} + 3 x_n
\end{equation*}

\subsubsection{Part (ii)}
Find the general formula for $ x_n $, $ n \geq 0 $.

\paragraph{Answer}
The characteristic polynomial is given by
\begin{equation*}
    P(z) = z^3 - 5 z^2 + 7 z - 3 = (z - 1)^2(z - 3),
\end{equation*}
with roots $ \lambda_1 = 1 $ and $ \lambda_2 = 3$.
The general form is
\begin{align*}
    x_n &= \sum_{j=1}^{2} \left( \sum_{i=0}^{m(\lambda_j)-1} C_{i,j} n^i \right) \lambda_j^n \\
        &= \lambda_1^n \sum_{i=0}^{1} C_{i,1} n^i + \lambda_2^n C_2 \\
        &= \lambda_1^n C_{0, 1} + \lambda_1^n C_{1, 1} n + \lambda_2^n C_2 \\
        &= C_1 + C_2 n + C_3 3^n
\end{align*}
Since $ x_0 = 1 $, $ x_1 = 5 $, and $ x_2 = 18 $, we find $ C_1 = -\frac{1}{2} $, $ C_2 = -\frac{5}{4} $, and $ C_3 = \frac{9}{4} $.
We conclude that
\begin{equation*}
    x_n = \frac{3^{n + 2} - 2n - 5}{4}
\end{equation*}

\subsection{Question 8}
Let $ P(z) = \sum_{i=0}^{k} a_i z^i $ be the characteristic polynomial corresponding to the linear recursion
\begin{equation*}
    \sum_{i=0}^{k} a_i x_{n + i} = 0, \quad \forall n \geq 0.
\end{equation*}
Assume that $ \lambda $ is a root of multiplicity 2 of $ P(z) $.
Show that the sequence $ (y_n)_{n \geq 0} $ given by
\begin{equation*}
    y_n = C n \lambda^n, \quad n \geq 0,
\end{equation*}
where $ C $ is an arbitrary constant, satisfies the recursion.

\paragraph{Answer}
\begin{align*}
    \sum_{i=0}^{k} a_i y_{n + i} &= \sum_{i=0}^{k} a_i C (n + i) \lambda^{n + i} \\
                                 &= C n \sum_{i=0}^{k} a_i \lambda^{n + i} + C \sum_{i=0}^{k} a_i i \lambda^{n + i} \\
                                 &= C n \lambda^n \sum_{i=0}^{k} a_i \lambda^i + C \lambda^{n + 1} \sum_{i=0}^{k} i a_i \lambda^{i - 1} \\
                                 &= C n \lambda^n P(\lambda) + C \lambda^{n + 1} P'(\lambda) \\
                                 &= 0.
\end{align*}

\subsection{Question 9}
Let $ n > 0 $. Show that
\begin{align*}
    O(x^n) + O(x^n) &= O(x^n), \quad \text{as} \quad x \rightarrow 0; \\
    o(x^n) + o(x^n) &= o(x^n), \quad \text{as} \quad x \rightarrow 0.
\end{align*}

\paragraph{Answer}
Let $ f(x) = O(x^n) $ and $ g(x) = O(x^n) $, then
\begin{equation*}
    \limsup_{x \rightarrow 0} \left| \frac{f(x)}{x^n} \right| < \infty \quad \text{and} \quad \limsup_{x \rightarrow 0} \left| \frac{g(x)}{x^n} \right| < \infty.
\end{equation*}
We see that
\begin{equation*}
    \limsup_{x \rightarrow 0} \left| \frac{f(x) + g(x)}{x^n} \right| \leq \limsup_{x \rightarrow 0} \left| \frac{f(x)}{x^n} \right| + \limsup_{x \rightarrow 0} \left| \frac{g(x)}{x^n} \right| < \infty,
\end{equation*}
and therefore $ O(x^n) + O(x^n) = O(x^n) $.

Let $ f(x) = o(x^n) $ and $ g(x) = o(x^n) $, then
\begin{equation*}
    \lim_{x \rightarrow 0} \left| \frac{f(x)}{x^n} \right| = 0 \quad \text{and} \quad \lim_{x \rightarrow 0} \left| \frac{g(x)}{x^n} \right| = 0.
\end{equation*}
We see that
\begin{equation*}
    \lim_{x \rightarrow 0} \left| \frac{f(x) + g(x)}{x^n} \right| \leq \lim_{x \rightarrow 0} \left| \frac{f(x)}{x^n} \right| + \lim_{x \rightarrow 0} \left| \frac{g(x)}{x^n} \right| = 0,
\end{equation*}
and therefore $ o(x^n) + o(x^n) = o(x^n) $.

\subsection{Question 10}
Prove that
\begin{align*}
    \sum_{k=1}^{n} k^2 &= O(n^3), \quad \text{as} \quad n \rightarrow \infty; \\
    \sum_{k=1}^{n} k^2 &= \frac{n^3}{3} + O(n^2), \quad \text{as} \quad n \rightarrow \infty,
\end{align*}
i.e., show that
\begin{equation*}
    \limsup_{n \rightarrow \infty} \frac{\sum_{k=1}^{n} k^2}{n^3} < \infty
\end{equation*}
and that
\begin{equation*}
    \limsup_{n \rightarrow \infty} \frac{\sum_{k=1}^{n} k^2 - \frac{n^3}{3}}{n^2} < \infty.
\end{equation*}
Similarly, prove that
\begin{align*}
    \sum_{k=1}^{n} k^3 &= O(n^4), \quad \text{as} \quad n \rightarrow \infty; \\
    \sum_{k=1}^{n} k^3 &= \frac{n^4}{4} + O(n^3), \quad \text{as} \quad n \rightarrow \infty,
\end{align*}

\paragraph{Answer}
Using \eqref{eq:sum-pow2}
\begin{align*}
    \limsup_{n \rightarrow \infty} \frac{\sum_{k=1}^{n} k^2}{n^3} &= \limsup_{n \rightarrow \infty} \frac{\frac{n (n + 1) (2 n + 1)}{6}}{n^3} \\
                                                                  &= \limsup_{n \rightarrow \infty} \frac{2 n^3 + 3 n^2 + n}{6 n^3} \\
                                                                  &= \frac{1}{3} < \infty \\
    \limsup_{n \rightarrow \infty} \frac{\sum_{k=1}^{n} k^2 - \frac{n^3}{3}}{n^2} &= \limsup_{n \rightarrow \infty} \frac{\frac{2 n^3 + 3 n^2 + n}{6} - \frac{n^3}{3}}{n^2} \\
                                                                                  &= \limsup_{n \rightarrow \infty} \frac{3 n^2 + n}{6 n^2} \\
                                                                                  &= \frac{1}{2} \\
                                                                                  &< \infty
\end{align*}
Using \eqref{eq:sum-pow3}
\begin{align*}
    \limsup_{n \rightarrow \infty} \frac{\sum_{k=1}^{n} k^3}{n^4} &= \limsup_{n \rightarrow \infty} \frac{\left( \frac{n (n + 1)}{2} \right)^2}{n^4} \\
                                                                  &= \limsup_{n \rightarrow \infty} \frac{n^2 (n^2 + 2n + 1)}{4 n^4} \\
                                                                  &= \frac{1}{4} \\
                                                                  &< \infty \\
    \limsup_{n \rightarrow \infty} \frac{\sum_{k=1}^{n} k^3 - \frac{n^4}{4}}{n^3} &= \limsup_{n \rightarrow \infty} \frac{ \frac{(n^4 + 2 n^3 + n^2)}{4} - \frac{n^4}{4}}{n^3} \\
                                                                  &= \limsup_{n \rightarrow \infty} \frac{2 n^3 + n^2}{4 n^3} \\
                                                                  &= \frac{1}{2} \\
                                                                  &< \infty
\end{align*}
