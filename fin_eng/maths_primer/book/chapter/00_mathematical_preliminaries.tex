\chapter{Mathematical preliminaries}

\section{Even and odd functions}

\begin{definition}
    The function \(f : \mathbb{R} \rightarrow \mathbb{R}\) is an even function iff
    \begin{equation}
        f(-x) = f(x), \quad \forall x \in \mathbb{R}.
    \end{equation}
\end{definition}
The graph of any even function is symmetric with respect to the \(y\)-axis.

\begin{lemma}
    Let \(f(x)\) be an integrable even function. Then,
    \begin{equation}
        \int_{-a}^{0} f(x) dx = \int_{0}^{a} f(x) dx, \quad \forall a \in \mathbb{R},
    \end{equation}
    and therefore
    \begin{equation}
        \int_{-a}^{a} f(x) dx = 2 \int_{0}^{a} f(x) dx, \quad \forall a \in \mathbb{R}.
    \end{equation}
    Moreover, if \(\int_{0}^{\infty} f(x) dx\) exists, then
    \begin{equation}
        \int_{-\infty}^{0} f(x) dx = \int_{0}^{\infty} f(x) dx,
    \end{equation}
    and
    \begin{equation}
        \int_{-\infty}^{\infty} f(x) dx = 2 \int_{0}^{\infty} f(x) dx.
    \end{equation}
\end{lemma}

\begin{definition}
    The function \(f : \mathbb{R} \rightarrow \mathbb{R}\) is an odd function iff
    \begin{equation}
        f(-x) = -f(x), \quad \forall x \in \mathbb{R}.
        \label{eq:definition:odd}
    \end{equation}
\end{definition}
If we let \(x = 0\) in \eqref{eq:definition:odd}, we find that \(f(0) = 0\) for any odd function \(f(x)\).
Also, the graph of any odd function is symmetric with respect to the point (0, 0).

\begin{lemma}
    Let \(f(x)\) be an integrable odd function. Then,
    \begin{equation}
        \int_{-a}^{a} f(x) dx = 0, \quad \forall a \in \mathbb{R}.
    \end{equation}
    Moreover, if \(\int_{0}^{\infty} f(x) dx\) exists, then
    \begin{equation}
        \int_{-\infty}^{\infty} f(x) dx = 0.
    \end{equation}
\end{lemma}

\section{Useful sums with interesting proofs}

The following sums occur frequently when estimating operation counts of numerical algorithms:
\begin{align}
    \sum_{k=1}^{n} k &= \frac{n (n + 1)}{2}; \label{eq:sum-pow1}\\
    \sum_{k=1}^{n} k^2 &= \frac{n (n + 1) (2 n + 1)}{6}; \label{eq:sum-pow2}\\
    \sum_{k=1}^{n} k^3 &= \left( \frac{n (n + 1)}{2} \right)^2. \label{eq:sum-pow3}
\end{align}

\section{Sequences satisfying linear recursions}

\begin{definition}
    A sequence \((x_n)_{n \geq 0}\) satisfies a linear recursion of order \(k\) iff there exist constants \(a_i\), \(i = 0 : k\) with \(a_k \neq 0\), such that
    \begin{equation}
        \sum_{i=0}^{k} a_i x_{n+i} = 0, \quad \forall n \geq 0.
        \label{eq:definition:linear-recursion}
    \end{equation}
\end{definition}

The recursion \eqref{eq:definition:linear-recursion} is called a linear recursion because of the following linearity properties:\\
(i) If the sequence \((x_n)_{n \geq 0}\) satisfies the linear recursion \eqref{eq:definition:linear-recursion}, then the sequence \((z_n)_{n \geq 0}\) given by
\begin{equation}
    z_n = C x_n, \quad \forall n \geq 0,
\end{equation}
where \(C\) is an arbitrary constant, also satisfies the linear recursion \eqref{eq:definition:linear-recursion}.\\
(ii) If the sequences \((x_n)_{n \geq 0}\) and \((y_n)_{n \geq 0}\) satisfies the linear recursion \eqref{eq:definition:linear-recursion}, then the sequence \((z_n)_{n \geq 0}\) given by
\begin{equation}
    z_n = x_n + y_n, \quad \forall n \geq 0,
\end{equation}
also satisfies the linear recursion \eqref{eq:definition:linear-recursion}.

\begin{definition}
    The characteristic polynomial \(P(z)\) corresponding to the linear recursion \(\sum_{i=0}^{k} a_i x_{n+i} = 0\), for all \(n \geq 0\), is defined as
    \begin{equation}
        P(z) = \sum_{i=0}^{k} a_i z^i.
    \end{equation}
\end{definition}
