\chapter{Improper integrals. Numerical integration. Interest rates. Bonds.}

\section{Double integrals}
Let $ D \subset \mathbb{R}^2 $ be a bounded region and let
    $ f : D \rightarrow \mathbb{R} $ be a continuous function.
The double integral of $ f $ over $ D $, denoted by
\begin{equation*}
    \int \int_D f,
\end{equation*}
represents the volume of the three dimensional body between the domain $ D $ in
    the two dimensional plane and the graph of the function $ f(x, y) $.

For simplicity, assume that the domain $ D $ is bounded and convex, i.e., for
    any two points $ x_1 $ and $ x_2 $ in $ D $, all the points on the segment
    joining $ x_1 $ and $ x_2 $ are in $ D $ as well.
Also, assume that there exist two continuous functions $ f_1(x) $ and $ f_2(x) $
    such that $ D $ can be described as follow:
\begin{equation}
    D = {(x, y) | a \leq x \leq b \quad \text{and} \quad f_1(x) \leq y \leq
        f_2(x)}.
    \label{eq:bounded-convex-domain}
\end{equation}
The functions $ f_1(x) $ and $ f_2(x) $ are well defined by
    \eqref{eq:bounded-convex-domain} since the domain $ D $ is bounded and
    convex.
Then, by definition,
\begin{equation}
    \int \int_d f(x, y) dy dx = \int_{a}^{b} \left( \int_{f_1(x)}^{f_2(x)}
        f(x, y) dy \right) dx.
    \label{eq:double-integral-yx}
\end{equation}

If there exists two continuous functions $ g_1(x) $ and $ g_2(x) $ such that
    $ D = {(x, y) | c \leq y \leq d\ \text{and}\ g_1(y) \leq x \leq g_2(x)} $,
    then, by definition,
\begin{equation}
    \int \int_D f(x, y) dx dy = \int_{c}^{d} \left( \int_{g_1(y)}^{g_2(y)}
        f(x, y) dx \right) dy.
    \label{eq:double-integral-xy}
\end{equation}

\begin{theorem}[Fubini's Theorem]
    With the notations above, if the function $ f(x, y) $ is continuous, then
        the integrals \eqref{eq:double-integral-yx} and
        \eqref{eq:double-integral-xy} are equal to each other and to the double
        integral of $ f(x, y) $ over $ D $, i.e., the order of integration does
        not matter:
    \begin{equation}
        \int \int_D f = \int \int_D f(x, y) dx dy = \int \int_D f(x, y) dy dx.
        \label{eq:theorem:fubini}
    \end{equation}
\end{theorem}

\subsection{Improper integrals}
We consider three types of improper integrals:

\paragraph{Type 1}
Integrate the function $ f(x) $ over an infinite interval of the form
    $ [a, \infty) $ or $ (-\infty, b] $.
The integral $ \int_{a}^{\infty} f(x) dx $ exists if and only if the limit as
    $ t \rightarrow \infty $ of the definite integral of $ f(x) $ between $ a $
    and $ t $ exists and is finite.
The integral $ \int_{-\infty}^{b} f(x) dx $ exists if and only if the limit as
    $ t \rightarrow -\infty $ of the definite integral of $ f(x) $ between $ t $
    and $ b $ exists and is finite.
Then
\begin{align*}
    \int_{a}^{\infty} f(x) dx &= \lim_{t \rightarrow \infty} \int_{a}^{t} f(x)
        dx; \\
    \int_{-\infty}^{b} f(x) dx &= \lim_{t \rightarrow -\infty} \int_{t}^{b}
        f(x) dx.
\end{align*}

Adding and subtracting improper integrals of this type follows rules similar to
    those for definite integrals:
\begin{lemma}
    Let $ f : \mathbb{R} \rightarrow \mathbb{R} $ be an integrable function over
        the interval $ [a, \infty) $.
    If $ b > a $, then $ f(x) $ is also integrable over the interval
        $ [b, \infty) $ and
    \begin{equation*}
        \int_{a}^{\infty} f(x) dx - \int_{b}^{\infty} f(x) dx =
            \int_{a}^{b} f(x) dx.
    \end{equation*}
    Let $ f(x) $ be an integrable function over the interval $ (-\infty, b] $.
    If $ a < b $, then $ f(x) $ is also integrable over the interval
        $ (-\infty, a] $ and
    \begin{equation*}
        \int_{-\infty}^{b} f(x) dx - \int_{-\infty}^{a} f(x) dx =
            \int_{a}^{b} f(x) dx.
    \end{equation*}
\end{lemma}

\paragraph{Type 2}
Integrate the function $ f(x) $ over an interval $ [a, b] $ where $ f(x) $ is
    unbounded as $ x $ approaches the end points $ a $ and/or $ b $.
For example, if the limit as $ x \searrow a $ of $ f(x) $ is infinite, then
    $ \int_{a}^{b} f(x) dx $ exists if and only if the limit as $ t searrow a $
    of the definite integral of $ f(x) $ between $ t $ and $ b $ exists and is
    finite, i.e.,
    \begin{equation*}
        \int_{a}^{b} f(x) dx = \lim_{x \searrow a} \int_{t}^{b} f(x) dx.
    \end{equation*}

\paragraph{Type 3}
Integrate the function $ f(x) $ on the entire real axis, i.e., on
    $ (-\infty, \infty) $.
The integral $ \int_{-\infty}^{\infty} f(x) dx $ exists if and only if a real
    number $ a $ such that both $ \int_{-\infty}^{a} f(x) dx $ and
    $ \int_{a}^{\infty} f(x) dx $ exist.
Then,
\begin{align}
    \int_{-\infty}^{\infty} f(x) dx
        &= \int_{-\infty}^{a} f(x) dx + \int_{a}^{\infty} f(x) dx \\
        &= \lim_{t_1 \rightarrow -\infty} \int_{t_1}^{a} f(x) dx +
            \lim_{t_2 \rightarrow \infty} \int_{a}^{t_2} f(x) dx.
            \label{eq:undefined-integral-real-axis}
\end{align}

It is incorrect to use, instead of \eqref{eq:undefined-integral-real-axis}, the
    following definition for the integral $ f(x) $ over the real axis
    $ (-\infty, \infty) $:
\begin{equation}
    \int_{-\infty}^{\infty} f(x) dx = \lim_{t \rightarrow \infty}
        \int_{t}^{t} f(x) dx.
    \label{eq:defined-integral-real-axis}
\end{equation}

However, if we know that the function $ f(x) $ is integrable over the entire
    real axis, then we can use formula \eqref{eq:defined-integral-real-axis} to
    evaluate it:
\begin{lemma}
    If the improper integral $ \int_{-\infty}^{\infty} f(x) dx $ exists, then
    \begin{equation}
        \int_{-\infty}^{\infty} f(x) dx = \lim_{t \rightarrow \infty}
            \int_{-t}^{t} f(x) dx.
        \label{eq:lemma:defined-integral-real-axis}
    \end{equation}
\end{lemma}

