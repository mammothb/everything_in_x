\chapter{Improper integrals. Numerical integration. Interest rates. Bonds.}

\section{Double integrals}
Let $ D \subset \mathbb{R}^2 $ be a bounded region and let
    $ f : D \rightarrow \mathbb{R} $ be a continuous function.
The double integral of $ f $ over $ D $, denoted by
\begin{equation*}
    \int \int_D f,
\end{equation*}
represents the volume of the three dimensional body between the domain $ D $ in
    the two dimensional plane and the graph of the function $ f(x, y) $.

For simplicity, assume that the domain $ D $ is bounded and convex, i.e., for
    any two points $ x_1 $ and $ x_2 $ in $ D $, all the points on the segment
    joining $ x_1 $ and $ x_2 $ are in $ D $ as well.
Also, assume that there exist two continuous functions $ f_1(x) $ and $ f_2(x) $
    such that $ D $ can be described as follow:
\begin{equation}
    D = {(x, y) | a \leq x \leq b \quad \text{and} \quad f_1(x) \leq y \leq
        f_2(x)}.
    \label{eq:bounded-convex-domain}
\end{equation}
The functions $ f_1(x) $ and $ f_2(x) $ are well defined by
    \eqref{eq:bounded-convex-domain} since the domain $ D $ is bounded and
    convex.
Then, by definition,
\begin{equation}
    \int \int_d f(x, y) dy dx = \int_{a}^{b} \left( \int_{f_1(x)}^{f_2(x)}
        f(x, y) dy \right) dx.
    \label{eq:double-integral-yx}
\end{equation}

If there exists two continuous functions $ g_1(x) $ and $ g_2(x) $ such that
    $ D = {(x, y) | c \leq y \leq d\ \text{and}\ g_1(y) \leq x \leq g_2(x)} $,
    then, by definition,
\begin{equation}
    \int \int_D f(x, y) dx dy = \int_{c}^{d} \left( \int_{g_1(y)}^{g_2(y)}
        f(x, y) dx \right) dy.
    \label{eq:double-integral-xy}
\end{equation}

\begin{theorem}[Fubini's Theorem]
    With the notations above, if the function $ f(x, y) $ is continuous, then
        the integrals \eqref{eq:double-integral-yx} and
        \eqref{eq:double-integral-xy} are equal to each other and to the double
        integral of $ f(x, y) $ over $ D $, i.e., the order of integration does
        not matter:
    \begin{equation}
        \int \int_D f = \int \int_D f(x, y) dx dy = \int \int_D f(x, y) dy dx.
        \label{eq:theorem:fubini}
    \end{equation}
\end{theorem}

\section{Improper integrals}
We consider three types of improper integrals:

\paragraph{Type 1}
Integrate the function $ f(x) $ over an infinite interval of the form
    $ [a, \infty) $ or $ (-\infty, b] $.
The integral $ \int_{a}^{\infty} f(x) dx $ exists if and only if the limit as
    $ t \rightarrow \infty $ of the definite integral of $ f(x) $ between $ a $
    and $ t $ exists and is finite.
The integral $ \int_{-\infty}^{b} f(x) dx $ exists if and only if the limit as
    $ t \rightarrow -\infty $ of the definite integral of $ f(x) $ between $ t $
    and $ b $ exists and is finite.
Then
\begin{align*}
    \int_{a}^{\infty} f(x) dx &= \lim_{t \rightarrow \infty} \int_{a}^{t} f(x)
        dx; \\
    \int_{-\infty}^{b} f(x) dx &= \lim_{t \rightarrow -\infty} \int_{t}^{b}
        f(x) dx.
\end{align*}

Adding and subtracting improper integrals of this type follows rules similar to
    those for definite integrals:
\begin{lemma}
    Let $ f : \mathbb{R} \rightarrow \mathbb{R} $ be an integrable function over
        the interval $ [a, \infty) $.
    If $ b > a $, then $ f(x) $ is also integrable over the interval
        $ [b, \infty) $ and
    \begin{equation*}
        \int_{a}^{\infty} f(x) dx - \int_{b}^{\infty} f(x) dx =
            \int_{a}^{b} f(x) dx.
    \end{equation*}
    Let $ f(x) $ be an integrable function over the interval $ (-\infty, b] $.
    If $ a < b $, then $ f(x) $ is also integrable over the interval
        $ (-\infty, a] $ and
    \begin{equation*}
        \int_{-\infty}^{b} f(x) dx - \int_{-\infty}^{a} f(x) dx =
            \int_{a}^{b} f(x) dx.
    \end{equation*}
\end{lemma}

\paragraph{Type 2}
Integrate the function $ f(x) $ over an interval $ [a, b] $ where $ f(x) $ is
    unbounded as $ x $ approaches the end points $ a $ and/or $ b $.
For example, if the limit as $ x \searrow a $ of $ f(x) $ is infinite, then
    $ \int_{a}^{b} f(x) dx $ exists if and only if the limit as $ t searrow a $
    of the definite integral of $ f(x) $ between $ t $ and $ b $ exists and is
    finite, i.e.,
    \begin{equation*}
        \int_{a}^{b} f(x) dx = \lim_{x \searrow a} \int_{t}^{b} f(x) dx.
    \end{equation*}

\paragraph{Type 3}
Integrate the function $ f(x) $ on the entire real axis, i.e., on
    $ (-\infty, \infty) $.
The integral $ \int_{-\infty}^{\infty} f(x) dx $ exists if and only if a real
    number $ a $ such that both $ \int_{-\infty}^{a} f(x) dx $ and
    $ \int_{a}^{\infty} f(x) dx $ exist.
Then,
\begin{align}
    \int_{-\infty}^{\infty} f(x) dx
        &= \int_{-\infty}^{a} f(x) dx + \int_{a}^{\infty} f(x) dx \\
        &= \lim_{t_1 \rightarrow -\infty} \int_{t_1}^{a} f(x) dx +
            \lim_{t_2 \rightarrow \infty} \int_{a}^{t_2} f(x) dx.
            \label{eq:undefined-integral-real-axis}
\end{align}

It is incorrect to use, instead of \eqref{eq:undefined-integral-real-axis}, the
    following definition for the integral $ f(x) $ over the real axis
    $ (-\infty, \infty) $:
\begin{equation}
    \int_{-\infty}^{\infty} f(x) dx = \lim_{t \rightarrow \infty}
        \int_{t}^{t} f(x) dx.
    \label{eq:defined-integral-real-axis}
\end{equation}

However, if we know that the function $ f(x) $ is integrable over the entire
    real axis, then we can use formula \eqref{eq:defined-integral-real-axis} to
    evaluate it:
\begin{lemma}
    If the improper integral $ \int_{-\infty}^{\infty} f(x) dx $ exists, then
    \begin{equation}
        \int_{-\infty}^{\infty} f(x) dx = \lim_{t \rightarrow \infty}
            \int_{-t}^{t} f(x) dx.
        \label{eq:lemma:defined-integral-real-axis}
    \end{equation}
\end{lemma}

\section{Differentiating improper integrals with respect to the integration
    limits}
\begin{lemma}
    Let $ f : \mathbb{R} \rightarrow \mathbb{R} $ be a continuous function such
        that the improper integral $ \int_{-\infty}^{\infty} f(x) dx $ exists.
    Let $ g, h : \mathbb{R} \rightarrow \mathbb{R} $ be given by
    \begin{equation*}
        g(t) = \int_{-\infty}^{b(t)} f(x) dx; \quad
        h(t) = \int_{a(t)}^{\infty} f(x) dx,
    \end{equation*}
    where $ a(t) $ and $ b(t) $ are differentiable functions.
    Then $ g(t) $ and $ h(t) $ are differentiable, and
    \begin{align*}
        g'(t) &= f(b(t)) b'(t); \\
        h'(t) &= -f(a(t)) a'(t).
    \end{align*}
\end{lemma}

\section{Numerical methods for computing definite integrals: Midpoint rule,
    Trapezoidal rule, and Simpson's rule}
Computing the value of a definite integral using the Fundamental Theorem of
    Calculus is not always possible.
The approximate values of the definite integral are computed using numerical
    integration methods in these cases.
We present three of the most common such methods.

Let $ f : [a, b] \rightarrow \mathbb{R} $ be an integrable function.
To compute an approximate value of the integral
\begin{equation*}
    I = \int_{a}^{b} f(x) dx,
\end{equation*}
we partition the interval $ [a, b] $ into $ n $ intervals of equal size
    $ h = \frac{b - a}{h} $ by using the nodes $ a_i = a + i h $, for
    $ i = 0 : n $, i.e.,
\begin{equation*}
    a = a_0 < a_1 < a_2 < ... < a_{n - 1} < a_n = b.
\end{equation*}
Note that $ a_i - a_{i - 1} = h $, $ i = 1 : n $.
Let $ x_i $ be the midpoint of the interval $ [a_{i - 1}, a_i] $, i.e.,
\begin{equation*}
    x_i = \frac{a_{i - 1} + a_i}{2}, \quad \forall i = 1 : n.
\end{equation*}

The integral $ I $ can be written as
\begin{equation}
    I = \sum_{i=1}^{n} \int_{a_{i-1}}^{a_i} f(x) dx.
    \label{eq:integral-interval}
\end{equation}
On each interval $ [a_{i-1}, a_i] $, $ i = 1 : n $, the function $ f(x) $ is
    approximated by a simpler function whose integral on $ [a_{i-1}, a_i] $ can
    be computed exactly.
The resulting values are summed up to obtain an approximate value of $ I $.
Depending on whether constant functions, linear functions, or quadratic
    functions are used to approximate $ f(x) $, the resulting numerical
    integration methods are called the Midpoint rule, the Trapezoidal rule, and
    the Simpson's rule, respectively.

\paragraph{Midpoint rule}
Approximate $ f(x) $ on the interval $ [a_{i-1}, a_i] $ by the constant function
    $ c_i(x) $ equal to the value of the function $ f $ at the midpoint $ x_i $
    of the interval $ [a_{i-1}, a_i] $, i.e.,
\begin{equation}
    c_i(x) = f(x_i), \quad \forall x \in [a_{i-1}, a_i].
    \label{eq:midpoint-constant-function}
\end{equation}
Then,
\begin{equation}
    \int_{a_{i-1}}^{a_i} f(x) dx \approx \int_{a_{i-1}}^{a_i} c_i(x) dx =
        (a_i - a_{i-1}) f(x_i) = h f(x_i).
    \label{eq:midpoint-integration-approximation}
\end{equation}
