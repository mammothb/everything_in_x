\chapter{Improper integrals. Numerical integration. Interest rates. Bonds.}

\section{Double integrals}
Let $ D \subset \mathbb{R}^2 $ be a bounded region and let
    $ f : D \rightarrow \mathbb{R} $ be a continuous function.
The double integral of $ f $ over $ D $, denoted by
\begin{equation*}
    \int \int_D f,
\end{equation*}
represents the volume of the three dimensional body between the domain $ D $ in
    the two dimensional plane and the graph of the function $ f(x, y) $.

For simplicity, assume that the domain $ D $ is bounded and convex, i.e., for
    any two points $ x_1 $ and $ x_2 $ in $ D $, all the points on the segment
    joining $ x_1 $ and $ x_2 $ are in $ D $ as well.
Also, assume that there exist two continuous functions $ f_1(x) $ and $ f_2(x) $
    such that $ D $ can be described as follow:
\begin{equation}
    D = {(x, y) | a \leq x \leq b \quad \text{and} \quad f_1(x) \leq y \leq
        f_2(x)}.
    \label{eq:bounded-convex-domain}
\end{equation}
The functions $ f_1(x) $ and $ f_2(x) $ are well defined by
    \eqref{eq:bounded-convex-domain} since the domain $ D $ is bounded and
    convex.
Then, by definition,
\begin{equation}
    \int \int_d f(x, y) dy dx = \int_{a}^{b} \left( \int_{f_1(x)}^{f_2(x)}
        f(x, y) dy \right) dx.
    \label{eq:double-integral-yx}
\end{equation}

If there exists two continuous functions $ g_1(x) $ and $ g_2(x) $ such that
    $ D = {(x, y) | c \leq y \leq d\ \text{and}\ g_1(y) \leq x \leq g_2(x)} $,
    then, by definition,
\begin{equation}
    \int \int_D f(x, y) dx dy = \int_{c}^{d} \left( \int_{g_1(y)}^{g_2(y)}
        f(x, y) dx \right) dy.
    \label{eq:double-integral-xy}
\end{equation}

\begin{theorem}[Fubini's Theorem]
    With the notations above, if the function $ f(x, y) $ is continuous, then
        the integrals \eqref{eq:double-integral-yx} and
        \eqref{eq:double-integral-xy} are equal to each other and to the double
        integral of $ f(x, y) $ over $ D $, i.e., the order of integration does
        not matter:
    \begin{equation}
        \int \int_D f = \int \int_D f(x, y) dx dy = \int \int_D f(x, y) dy dx.
        \label{eq:theorem:fubini}
    \end{equation}
\end{theorem}

\section{Improper integrals}
We consider three types of improper integrals:

\paragraph{Type 1}
Integrate the function $ f(x) $ over an infinite interval of the form
    $ [a, \infty) $ or $ (-\infty, b] $.
The integral $ \int_{a}^{\infty} f(x) dx $ exists if and only if the limit as
    $ t \rightarrow \infty $ of the definite integral of $ f(x) $ between $ a $
    and $ t $ exists and is finite.
The integral $ \int_{-\infty}^{b} f(x) dx $ exists if and only if the limit as
    $ t \rightarrow -\infty $ of the definite integral of $ f(x) $ between $ t $
    and $ b $ exists and is finite.
Then
\begin{align*}
    \int_{a}^{\infty} f(x) dx &= \lim_{t \rightarrow \infty} \int_{a}^{t} f(x)
        dx; \\
    \int_{-\infty}^{b} f(x) dx &= \lim_{t \rightarrow -\infty} \int_{t}^{b}
        f(x) dx.
\end{align*}

Adding and subtracting improper integrals of this type follows rules similar to
    those for definite integrals:
\begin{lemma}
    Let $ f : \mathbb{R} \rightarrow \mathbb{R} $ be an integrable function over
        the interval $ [a, \infty) $.
    If $ b > a $, then $ f(x) $ is also integrable over the interval
        $ [b, \infty) $ and
    \begin{equation*}
        \int_{a}^{\infty} f(x) dx - \int_{b}^{\infty} f(x) dx =
            \int_{a}^{b} f(x) dx.
    \end{equation*}
    Let $ f(x) $ be an integrable function over the interval $ (-\infty, b] $.
    If $ a < b $, then $ f(x) $ is also integrable over the interval
        $ (-\infty, a] $ and
    \begin{equation*}
        \int_{-\infty}^{b} f(x) dx - \int_{-\infty}^{a} f(x) dx =
            \int_{a}^{b} f(x) dx.
    \end{equation*}
\end{lemma}

\paragraph{Type 2}
Integrate the function $ f(x) $ over an interval $ [a, b] $ where $ f(x) $ is
    unbounded as $ x $ approaches the end points $ a $ and/or $ b $.
For example, if the limit as $ x \searrow a $ of $ f(x) $ is infinite, then
    $ \int_{a}^{b} f(x) dx $ exists if and only if the limit as $ t searrow a $
    of the definite integral of $ f(x) $ between $ t $ and $ b $ exists and is
    finite, i.e.,
    \begin{equation*}
        \int_{a}^{b} f(x) dx = \lim_{x \searrow a} \int_{t}^{b} f(x) dx.
    \end{equation*}

\paragraph{Type 3}
Integrate the function $ f(x) $ on the entire real axis, i.e., on
    $ (-\infty, \infty) $.
The integral $ \int_{-\infty}^{\infty} f(x) dx $ exists if and only if a real
    number $ a $ such that both $ \int_{-\infty}^{a} f(x) dx $ and
    $ \int_{a}^{\infty} f(x) dx $ exist.
Then,
\begin{align}
    \int_{-\infty}^{\infty} f(x) dx
        &= \int_{-\infty}^{a} f(x) dx + \int_{a}^{\infty} f(x) dx \\
        &= \lim_{t_1 \rightarrow -\infty} \int_{t_1}^{a} f(x) dx +
            \lim_{t_2 \rightarrow \infty} \int_{a}^{t_2} f(x) dx.
            \label{eq:undefined-integral-real-axis}
\end{align}

It is incorrect to use, instead of \eqref{eq:undefined-integral-real-axis}, the
    following definition for the integral $ f(x) $ over the real axis
    $ (-\infty, \infty) $:
\begin{equation}
    \int_{-\infty}^{\infty} f(x) dx = \lim_{t \rightarrow \infty}
        \int_{t}^{t} f(x) dx.
    \label{eq:defined-integral-real-axis}
\end{equation}

However, if we know that the function $ f(x) $ is integrable over the entire
    real axis, then we can use formula \eqref{eq:defined-integral-real-axis} to
    evaluate it:
\begin{lemma}
    If the improper integral $ \int_{-\infty}^{\infty} f(x) dx $ exists, then
    \begin{equation}
        \int_{-\infty}^{\infty} f(x) dx = \lim_{t \rightarrow \infty}
            \int_{-t}^{t} f(x) dx.
        \label{eq:lemma:defined-integral-real-axis}
    \end{equation}
\end{lemma}

\section{Differentiating improper integrals with respect to the integration
    limits}
\begin{lemma}
    Let $ f : \mathbb{R} \rightarrow \mathbb{R} $ be a continuous function such
        that the improper integral $ \int_{-\infty}^{\infty} f(x) dx $ exists.
    Let $ g, h : \mathbb{R} \rightarrow \mathbb{R} $ be given by
    \begin{equation*}
        g(t) = \int_{-\infty}^{b(t)} f(x) dx; \quad
        h(t) = \int_{a(t)}^{\infty} f(x) dx,
    \end{equation*}
    where $ a(t) $ and $ b(t) $ are differentiable functions.
    Then $ g(t) $ and $ h(t) $ are differentiable, and
    \begin{align*}
        g'(t) &= f(b(t)) b'(t); \\
        h'(t) &= -f(a(t)) a'(t).
    \end{align*}
\end{lemma}

\section{Numerical methods for computing definite integrals: Midpoint rule,
    Trapezoidal rule, and Simpson's rule}
Computing the value of a definite integral using the Fundamental Theorem of
    Calculus is not always possible.
The approximate values of the definite integral are computed using numerical
    integration methods in these cases.
We present three of the most common such methods.

Let $ f : [a, b] \rightarrow \mathbb{R} $ be an integrable function.
To compute an approximate value of the integral
\begin{equation*}
    I = \int_{a}^{b} f(x) dx,
\end{equation*}
we partition the interval $ [a, b] $ into $ n $ intervals of equal size
    $ h = \frac{b - a}{h} $ by using the nodes $ a_i = a + i h $, for
    $ i = 0 : n $, i.e.,
\begin{equation*}
    a = a_0 < a_1 < a_2 < ... < a_{n - 1} < a_n = b.
\end{equation*}
Note that $ a_i - a_{i - 1} = h $, $ i = 1 : n $.
Let $ x_i $ be the midpoint of the interval $ [a_{i - 1}, a_i] $, i.e.,
\begin{equation*}
    x_i = \frac{a_{i - 1} + a_i}{2}, \quad \forall i = 1 : n.
\end{equation*}

The integral $ I $ can be written as
\begin{equation}
    I = \sum_{i=1}^{n} \int_{a_{i-1}}^{a_i} f(x) dx.
    \label{eq:integral-interval}
\end{equation}
On each interval $ [a_{i-1}, a_i] $, $ i = 1 : n $, the function $ f(x) $ is
    approximated by a simpler function whose integral on $ [a_{i-1}, a_i] $ can
    be computed exactly.
The resulting values are summed up to obtain an approximate value of $ I $.
Depending on whether constant functions, linear functions, or quadratic
    functions are used to approximate $ f(x) $, the resulting numerical
    integration methods are called the Midpoint rule, the Trapezoidal rule, and
    the Simpson's rule, respectively.

\paragraph{Midpoint rule}
Approximate $ f(x) $ on the interval $ [a_{i-1}, a_i] $ by the constant function
    $ c_i(x) $ equal to the value of the function $ f $ at the midpoint $ x_i $
    of the interval $ [a_{i-1}, a_i] $, i.e.,
\begin{equation}
    c_i(x) = f(x_i), \quad \forall x \in [a_{i-1}, a_i].
    \label{eq:midpoint-constant-function}
\end{equation}
Then,
\begin{equation}
    \int_{a_{i-1}}^{a_i} f(x) dx \approx \int_{a_{i-1}}^{a_i} c_i(x) dx =
        (a_i - a_{i-1}) f(x_i) = h f(x_i).
    \label{eq:midpoint-integration-approximation}
\end{equation}
From \eqref{eq:integral-interval} and
    \eqref{eq:midpoint-integration-approximation}, we obtain that the Midpoint
    Rule approximation $ I_n^M $ of $ I $ corresponding to $ n $ partition
    intervals is
\begin{align}
    \begin{split}
        I_n^M &= \sum_{i=1}^{n} \int_{a_{i-1}}^{a_i} c_i(x) dx \\
              &=  h \sum_{i=1}^{n} f(x_i).
        \label{eq:midpoint-rule-sum}
    \end{split}
\end{align}

\paragraph{Trapezoidal Rule}
Approximate $ f(x) $ on the interval $ [a_{i-1}, a_i] $ by the linear function
    $ l_i(x) $ equal to $ f(x) $ at the end points $ a_{i-1} $ and $ a_i $,
    i.e.,
\begin{equation*}
    l_i(a_{i-1}) = f(a_{i-1}) \quad \text{and} \quad l_i(a_i) = f(a_i).
\end{equation*}
By linear interpolation, it is easy to see that
\begin{equation}
    l_i(x) = \frac{x - a_{i-1}}{a_i - a_{i-1}} f(a_i) +
        \frac{a_i - x}{a_i - a_{i-1}} f(a_{i-1}), \quad
        \forall x \in [a_{i-1}, a_i].
    \label{eq:trapezoidal-linear-function}
\end{equation}
Then,
\begin{equation}
    \int_{a_{i-1}}^{a_i} f(x) dx \approx \int_{a_{i_1}}^{a_i} l_i(x) dx =
        \frac{h}{2} (f(a_{i-1}) + f(a_i)).
    \label{eq:trapezoidal-rule-approximation}
\end{equation}
From \eqref{eq:integral-interval} and \eqref{eq:trapezoidal-rule-approximation},
    we obtain that the Trapezoidal Rule approximation $ I_n^T $ of $ I $
    corresponding to $ n $ partition intervals is
\begin{align}
    \begin{split}
        I_n^T
            &= \sum_{i=1}^{n} \int_{a_{i-1}}^{a_i} l_i(x) dx \\
            &= \frac{h}{2} \left( f(a_0) + 2 \sum_{i=1}^{n-1} f(a_i) +
                f(a_n) \right)
        \label{eq:trapezoidal-rule-sum}
    \end{split}
\end{align}

\paragraph{Simpson's Rule}
Approximate $ f(x) $ on the interval $ [a_{i-1}, a_i] $ by the quadratic
    function $ q_i(x) $ equal to $ f(x) $ at $ a_{i-1} $, $ a_i $, and at the
    midpoint $ x_i = \frac{a_{i-1} - a_i}{2} $, i.e.,
\begin{equation*}
    q_i(a_{i-1}) = f(a_{i-1}); q_i(x_i) = f(x_i) \quad \text{and} \quad
        q_i(a_i) = f(a_i).
\end{equation*}
By quadratic interpolation, we find that
\begin{align}
    \begin{split}
        q_i(x)
            =& \frac{(x - a_{i-1})(x - x_i)}{(a_i - a_{i-1})(a_i - x_i)}
                f(a_i) + \frac{(a_i - x)(x - a_{i-1}}{(a_i - x_i)(x_i -
                a_{i-1})} f(x_i) \\
             &+ \frac{(a_i - x)(x_i - x)}{(a_i - a_{i-1})} f(a_{i-1}), \quad
                \forall x \in [a_{i-1}, a_i].
        \label{eq:simpsons-quadratic-function}
    \end{split}
\end{align}
Then,
\begin{equation}
    \int_{a_{i-1}}^{a_i} f(x) dx \approx \int_{a_{i-1}}^{a_i} q_i(x) dx =
        \frac{h}{6} (f(a_{i-1}) + 4 f(x_i) + f(a_i)).
    \label{eq:simpsons-rule-approximation}
\end{equation}
From \eqref{eq:integral-interval} and \eqref{eq:simpsons-rule-approximation},
    we obtain that the Simpson's Rule approximation $ I_n^S $ of $ I $
    corresponding to $ n $ partition intervals is
\begin{align}
    \begin{split}
        I_n^S
            &= \sum_{i=1}^{n} \int_{a_{i-1}}^{a_i} q_i(x) dx \\
            &= \frac{h}{6} \left( f(a_0) + 2 \sum_{i=1}^{n-1} f(a_i) + f(a_n) +
                4 \sum_{i=1}^{n} f(x_i) \right).
        \label{eq:simpsons-rule-sum}
    \end{split}
\end{align}

\section{Convergence of the Midpoint, Trapezoidal, and Simpson's rules}

We derived the formulas \eqref{eq:midpoint-rule-sum},
    \eqref{eq:trapezoidal-rule-sum}, and \eqref{eq:simpsons-rule-sum} for
    computing approximate values $ I_n^M $, $ I_n^T $, and $ I_n^S $ of the
    integral
\begin{equation*}
    I = \int_{a}^{b} f(x) dx
\end{equation*}
corresponding to the Midpoint, Trapezoidal, and Simpson's rules, respectively.
In this section, we discuss the convergence of these methods.

\begin{definition}
    Denote by $ I_n $ the approximation of $ I $ obtained using a numerical
        integration method with $ n $ partition integrals.
    The method is convergent if and only if the approximation $ I_n $ converge
        to I as the number of intervals $ n $ goes to infinity (and therefore
        as $ h = \frac{b - a}{h} $ goes to 0), i.e.,
    \begin{equation*}
        \lim_{n \rightarrow \infty} \lvert I - I_n \rvert = 0.
    \end{equation*}
    The order of convergence of the numerical integration method is $ k > 0 $
        if and only if
    \begin{equation*}
        \lvert I - I_n \rvert = O(h^k) = O \left( \frac{1}{n^k} \right).
    \end{equation*}
\end{definition}

\begin{theorem}
    Let $ I = \int_{a}^{b} f(x) dx $, and let $ I_n^M $, $ I_n^T $, or
        $ I_n^S $ be the approximations of $ I $ given by the Midpoint,
        Trapezoidal, and Simpson's rules corresponding to $ n $ partition
        intervals of size $ h = \frac{b - a}{n} $.

    (i) If $ f''(x) $ exists and is continuous on $ [a, b] $, then the
        approximation errors of the Midpoint and Trapezoidal rules can be
        bounded from above as follows:
    \begin{align}
        \lvert I - I_n^M \rvert &\leq \frac{h^2}{24} (b - a)
            \max_{a \leq x \leq b} \lvert f''(x) \rvert;
            \label{eq:midpoint-rule-upper-bound} \\
        \lvert I - I_n^T \rvert &\leq \frac{h^2}{12} (b - a)
            \max_{a \leq x \leq b} \lvert f''(x) \rvert.
            \label{eq:trapezoidal-rule-upper-bound}
    \end{align}
    Thus, the Midpoint and Trapezoidal rules are quadratically convergent, i.e.,
    \begin{align}
        \lvert I - I_n^M \rvert &= O \left( \frac{1}{n^2} \right), \quad
            \text{as}\quad n \rightarrow \infty; \\
        \lvert I - I_n^T \rvert &= O \left( \frac{1}{n^2} \right), \quad
            \text{as}\quad n \rightarrow \infty. \\
    \end{align}

    (ii) If $ f^{(4)}(x) $ exists and is continuous on $ [a, b] $, then
    \begin{equation}
        \lvert I - I_n^S \rvert \leq \frac{h^4}{2880} (b - a)
            \max_{a \leq x \leq b} \lvert f^{(4)}(x) \rvert,
        \label{eq:simpsons-rule-upper-bound}
    \end{equation}
    and Simpson's rule is fourth order convergent, i.e.,
    \begin{equation}
        \lvert I - I_n^S \rvert  O \left( \frac{1}{n^4} \right), \quad
            \text{as}\quad n \rightarrow \infty.
    \end{equation}
\end{theorem}

The upper bounds \eqref{eq:midpoint-rule-upper-bound},
    \eqref{eq:trapezoidal-rule-upper-bound}, and
    \eqref{eq:simpsons-rule-upper-bound} can be established using the following
    approximation error results: For any $ i = 1 : n $, there exist points
    $ \xi_{i,T} $, $ \xi_{i,M} $, and $ \xi_{i,S} $ in the interval
    $ (a_{i-1}, a_i) $ such that
\begin{align}
    \int_{a_{i-1}}^{a_i} f(x) dx - \int_{a_{i-1}}^{a_i} c_i(x) dx &=
        \frac{h^3}{24} f''(\xi_{i,T}); \\
    \int_{a_{i-1}}^{a_i} f(x) dx - \int_{a_{i-1}}^{a_i} l_i(x) dx &=
        -\frac{h^3}{12} f''(\xi_{i,M}); \\
    \int_{a_{i-1}}^{a_i} f(x) dx - \int_{a_{i-1}}^{a_i} q_i(x) dx &=
        -\frac{h^5}{2880} f^{(4)}(\xi_{i,S}),
\end{align}
where $ c_i(x) $, $ l_i(x) $, and $ q_i(x) $ are given by
    \eqref{eq:midpoint-constant-function},
    \eqref{eq:trapezoidal-linear-function}, and
    \eqref{eq:simpsons-quadratic-function}, respectively.

\subsection{Implementation of numerical integration methods}
Computing approximate values of the definite integral of a given function
    $ f(x) $ on the interval $ [a, b] $ using the Midpoint, Trapezoidal, or
    Simpson's rules requires implementation of formulas
    \eqref{eq:midpoint-rule-sum}, \eqref{eq:trapezoidal-rule-sum}, and
    \eqref{eq:simpsons-rule-sum}, i.e.,
\begin{align}
    I_n^M &= h \sum_{i=1}^{n} f(x_i); \\
    I_n^T &= h \left( \frac{f(a_0)}{2} + \frac{f(a_n)}{2} \right) +
        h \sum_{i=1}^{n-1} f(a_i); \\
    I_n^S &= h \left( \frac{f(a_0)}{6} + \frac{f(a_n)}{6} \right) +
        \frac{h}{3} \sum_{i=1}^{n-1} f(a_i) +
        \frac{2h}{3} \sum_{i=1}^{n} f(x_i). \\
\end{align}
Here, $ h = \frac{b-a}{n} $, $ a_i = a + i h $, $ i = 0:n $, and
    $ x_i = a + \left( i - \frac{1}{2} \right) h $, $ i = 1:n $.

A routine f\_int($ x $) evaluating the function to be integrated at the point
    $ x $ is required.
The end points $ a $ and $ b $ of the integration interval and the number of
    intervals $ n $ must also be specified.

\begin{algorithm}
    \caption{Pseudocode for Midpoint Rule}
    \begin{algorithmic}
        \State{Input:}
        \State{$ a $ = left endpoint of the integration interval}
        \State{$ b $ = right endpoint of the integration interval}
        \State{$ n $ = number of partition interval}
        \State{f\_int($ x $) = routine evaluating $ f(x) $}

        \State{}

        \State{Output:}
        \State{I\_midpoint = Midpoint Rule approximation of $ \int_{a}^{b} f(x) $}

        \State{}

        \State{$ h = (b - a) / n $; I\_midpoint = 0}
        \For{$ i = 1 : n $}
            \State{I\_midpoint = I\_midpoint + f\_int($ a + (i - 1 / 2) h $)}
        \EndFor
        \State{I\_midpoint = h $ \cdot $ I\_midpoint}
    \end{algorithmic}
\end{algorithm}

\begin{algorithm}
    \caption{Pseudocode for Trapezoidal Rule}
    \begin{algorithmic}
        \State{Input:}
        \State{$ a $ = left endpoint of the integration interval}
        \State{$ b $ = right endpoint of the integration interval}
        \State{$ n $ = number of partition interval}
        \State{f\_int($ x $) = routine evaluating $ f(x) $}

        \State{}

        \State{Output:}
        \State{I\_trap = Trapezoidal Rule approximation of
            $ \int_{a}^{b} f(x) $}

        \State{}

        \State{$ h = (b - a) / n $; I\_midpoint = 0}
        \State{I\_trap = f\_int(a) / 2 + f\_int(b) / 2}
        \For{$ i = 1 : (n-1) $}
            \State{I\_trap = I\_trap + f\_int($ a + i h $)}
        \EndFor
        \State{I\_trap = h $ \cdot $ I\_trap}
    \end{algorithmic}
\end{algorithm}

\begin{algorithm}
    \caption{Pseudocode for Simpson's Rule}
    \begin{algorithmic}
        \State{Input:}
        \State{$ a $ = left endpoint of the integration interval}
        \State{$ b $ = right endpoint of the integration interval}
        \State{$ n $ = number of partition interval}
        \State{f\_int($ x $) = routine evaluating $ f(x) $}

        \State{}

        \State{Output:}
        \State{I\_simpson = Simpson's Rule approximation of
            $ \int_{a}^{b} f(x) $}

        \State{}

        \State{$ h = (b - a) / n $; I\_midpoint = 0}
        \State{I\_simpson = f\_int(a) / 6 + f\_int(b) / 6}
        \For{$ i = 1 : (n-1) $}
            \State{I\_simpson = I\_simpson + f\_int($ a + i h $) / 3}
        \EndFor
        \For{$ i = 1 : n $}
            \State{I\_simpson = I\_simpson + 2
                f\_int($ a + (i - 1 / 2) h $) / 3}
        \EndFor
        \State{I\_simpson = h $ \cdot $ I\_simpson}
    \end{algorithmic}
\end{algorithm}

In practice, we want to find an approximate value that is within a prescribed
    tolerance \textit{tol} of the integral $ I $ of a given function $ f(x) $
    over the interval $ [a, b] $.
Simply using a numerical integration method with $ n $ partition intervals
    cannot work effectively, since we do not know in advance how large $ n $
    should be chosen to obtain an approximation of $ I $ with the desired
    accuracy.

We choose an integration method and smaller number of intervals, e.g., 4 or 8
    intervals, and compute the numerical approximation of the integral.
We then double the number of intervals and compute another approximation $ I $.
If the absolute value of the difference between the new and old approximations
    is smaller than the required tolerance \textit{tol}, we declare the last
    computed approximation of the integral to be the approximate value of $ I $
    that we are looking for.
Otherwise, double the number of intervals again and repeat the process until two
    consecutive numerical integration approximations are within the desired
    tolerance \textit{tol} of each other.
This condition is called the stopping criterion for the algorithm, and can be
    written formally as
\begin{equation}
    \lvert I_{new} - I_{old} \rvert < tol,
    \label{eq:stopping-criterion}
\end{equation}
where $ I_{old} $ and $ I_{new} $ are the last two approximations of $ I $ that
    were computed.
The pseudocode for this method is given in Algorithm
    \ref{alg:stopping-criterion}.
\begin{algorithm}
    \caption{Pseudocode for computing an approximate value of an integral with
        given tolerance}
    \label{alg:stopping-criterion}
    \begin{algorithmic}
        \State{Input:}
        \State{\textit{tol} = prescribed tolerance}
        \State{I\_numerical($ n $) = result of the numerical integration rule
            with $ n $ intervals; any integration rule can be used}

        \State{}

        \State{Output:}
        \State{I\_approx = approximation of $ \int_{a}^{b} f(x) $ with
            tolerance \textit{tol}}

        \State{}

        \State{$ n = 4 $; I\_old = I\_numerical($ n $)}
        \Comment{4 intervals initial partition}
        \State{$ n = 2n $; I\_new = I\_numerical($ n $)}
        \While{abs(I\_new - I\_old) $ > $ \textit{tol}}
            \State{I\_old = I\_new}
            \State{$ n = 2 n $}
            \State{I\_new = I\_numerical($ n $)}
        \EndWhile
        \State{I\_approx = I\_new}
    \end{algorithmic}
\end{algorithm}

\subsection{A concrete example}
We want to find an approximate value for
\begin{equation*}
    I = \int_{0}^{2} e^{-x^2} dx
\end{equation*}
which is within $ 0.5\ 10^{-7} $ of $ I $.

We implement the algorithm from Algorithm \ref{alg:stopping-criterion} for each
    of the numerical integration methods to compare their convergence
    properties.
We choose $ tol = 0.5\ 10^{-7} $.
For an initial partition of the interval $ [0, 2] $ into $ n = 4 $ intervals,
    the following approximate values of $ I $ are found using the Midpoint,
    Trapezoidal, and Simpson's rules, respectively:
\begin{equation*}
    I_4^M = 0.88278895;\quad I_4^T = 0.88061863;\quad I_4^S = 0.88206551.
\end{equation*}

Then, we double the number of partition intervals and compute the numerical
    approximates corresponding to each method.
We keep doubling the number of partition intervals until the stopping criterion
    \eqref{eq:stopping-criterion} is satisfied.
The results are recorded below:
\begin{table}
    \center
    \begin{tabular}[c]{l|l|l|l}
        \hline
        \multicolumn{1}{c|}{No. Intervals} &
        \multicolumn{1}{c}{Midpoint Rule} &
        \multicolumn{1}{c}{Trapezoidal Rule} &
        \multicolumn{1}{c}{Simpson's Rule} \\
        \hline
        4 & 0.88278895 & 0.88061863 & 0.88206551 \\
        8 & 0.88226870 & 0.88170379 & 0.88208040 \\
        16 & 0.88212887 & 0.88198624 & 0.88208133 \\
        32 & 0.88209330 & 0.88205756 & 0.88208139 \\
        64 & 0.88208437 & 0.88207543 & \\
        128 & 0.88208214 & 0.88207990 & \\
        256 & 0.88208158 & 0.88208102 & \\
        512 & 0.88208144 & 0.88208130 & \\
        \hline
    \end{tabular}
\end{table}

\section{Interest Rate Curves. Zero rates and instantaneous rates}
