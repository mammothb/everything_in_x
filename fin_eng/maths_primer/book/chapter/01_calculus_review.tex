\chapter{Calculus review. Plain vanilla options.}

\section{Brief review of differentiation}
The function $ f : \mathbb{R} \rightarrow \mathbb{R} $ is differentiable at the point $ x \in \mathbb{R} $ if the limit
\begin{equation*}
    \lim_{h \rightarrow 0} \frac{f(x + h) - f(x)}{h}
\end{equation*}
exists, in which case the derivative $ f'(x) $ is defined as
\begin{equation}
    f'(x) = \lim_{h \rightarrow 0} \frac{f(x + h) - f(x)}{h}.
    \label{eq:definition:derivative}
\end{equation}
\begin{theorem}[Product Rule.]
    The product $ f(x) g(x) $ of two differentiable functions $ f(x) $ and $ g(x) $ is differentiable, and
    \begin{equation}
        (f(x) g(x))' = f'(x) g(x) + f(x) g'(x).
        \label{eq:theorem:product-rule}
    \end{equation}
\end{theorem}
\begin{theorem}[Quotient Rule.]
    The quotient $ \frac{f(x)}{g(x)} $ of two differentiable functions $ f(x) $ and $ g(x) $ is differentiable at every point $ x $ where the function $ \frac{f(x)}{g(x)} $ is well defined, and
    \begin{equation}
        \left( \frac{f(x)}{g(x)} \right)' = \frac{f'(x) g(x) - f(x) g'(x)}{(g(x))^2}.
        \label{eq:definition:quotient-rule}
    \end{equation}
\end{theorem}
\begin{theorem}[Chain Rule.]
    The composite function $ (g \circ f)(x) = g(f(x)) $ of two differentiable functions $ f(x) $ and $ g(x) $ is differentiable at every point $ x $ where $ g(f(x)) $ is well defined, and
    \begin{equation}
        (g(f(x)))' = g'(f(x)) f'(x).
        \label{eq:definition:chain-rule}
    \end{equation}
\end{theorem}
The Chain Rule formula \eqref{eq:definition:chain-rule} can also be written as
\begin{equation*}
    \frac{dg}{dx} = \frac{dg}{du} \frac{du}{dx},
\end{equation*}
where $ u = f(x) $ is a function of $ x $ and $ g = g(u) = g(f(x)) $.

Chain Rule is often used for power functions, exponential functions, and logarithmic function:
\begin{align}
    \frac{d}{dx} ((f(x))^n) &= n (f(x))^{n - 1} f'(x); \label{eq:formula:chain-rule:power} \\
    \frac{d}{dx} (e^{f(x)}) &= e^{f(x)} f'(x); \label{eq:formula:chain-rule:exp} \\
    \frac{d}{dx} (\ln f(x)) &= \frac{f'(x)}{f(x)}. \label{eq:formula:chain-rule:log}
\end{align}

\begin{lemma}
    Let $ f : [a, b] \rightarrow [c, d] $ be a differentiable function, and assume that $ f(x) $ has an inverse function denoted by $ f^{-1}(x) $ with $ f^{-1} : [c, d] \rightarrow [a, b] $.
    The function $ f^{-1}(x) $ is differentiable at every point $ x \in [c, d] $ where $ f'(f^{-1}(x)) \neq 0 $ and
    \begin{equation}
        (f^{-1}(x))' = \frac{1}{f'(f^{-1}(x))}.
        \label{eq:lemma:inverse}
    \end{equation}
\end{lemma}

\section{Brief review of integration}
Let $ f : \mathbb{R} \rightarrow \mathbb{R} $ be an integrable function.
Recall that $ F(x) $ is the antiderivative of $ f(x) $ iff $ F'(x) = f $, i.e.,
\begin{equation*}
    F(x) = \int f(x) dx \quad \Longleftrightarrow \quad F'(x) = f(x).
\end{equation*}
\begin{theorem}[Fundamental Theorem of Calculus.]
    Let $ f(x) $ be a continuous function on the interval $ [a, b] $, and let $ F(x) $ be the antiderivative of $ f(x) $.
    Then
    \begin{equation*}
        \int_{a}^{b} f(x) dx = F(x) |_a^b = F(b) - F(a).
    \end{equation*}
\end{theorem}
Integration by parts is the counterpart for integration of the product rule.
\begin{theorem}[Integration by parts.]
    Let $ f(x) $ and $ g(x) $ be continuous function.
    Then
    \begin{equation}
        \int f(x) g(x) dx = F(x) g(x) - \int F(x) g'(x) dx,
        \label{eq:theorem:int-by-parts}
    \end{equation}
    where $ F(x) = \int f(x) dx $ is the antiderivative of $ f(x) $.
    For definite integrals,
    \begin{equation}
        \int_{a}^{b} f(x) g(x) dx = F(b) g(b) - F(a) g(a) - \int_{a}^{b} F(x) g'(x) dx.
        \label{eq:theorem:int-by-parts-definite}
    \end{equation}
\end{theorem}
Integration by substitution if the counterpart for integration of the chain rule.
\begin{theorem}[Integration by substitution]
    Let $ f(x) $ be an integrable function.
    Assume that $ g(u) $ is an invertible and continuously differentiable function.
    The substitution $ x = g(u) $ changes the integration variable from $ x $ to $ u $ as follows:
    \begin{equation}
        \int f(x) dx = \int f(g(u)) g'(u) du.
        \label{eq:theorem:int-by-sub}
    \end{equation}
    For definite integrals,
    \begin{equation}
        \int_{a}^{b} f(x) dx = \int_{g^{-1}(a)}^{g^{-1}(b)} f(g(u)) g'(u) du.
        \label{eq:theorem:int-by-sub-definite}
    \end{equation}
\end{theorem}

\section{Differentiating definite integrals}
If a definite integral has functions as limits of integration, e.g.,
\begin{equation*}
    \int_{a(t)}^{b(t)} f(x) dx,
\end{equation*}
or if the function to be integrated is a function of the integrating variable and of another variable, e.g.,
\begin{equation*}
    \int_{a}^{b} f(x, t) dx
\end{equation*}
then the result of the integration is a function (of the variable $ t $ in both cases above).

\begin{lemma}\label{lemma:int-function-limit}
    Let $ f : \mathbb{R} \rightarrow \mathbb{R} $ be a continuous function.
    Then,
    \begin{equation}
        \frac{d}{dt} \left( \int_{a(t)}^{b(t)} f(x) dx \right) = f(b(t)) b'(t) - f(a(t)) a'(t),
        \label{eq:lemma:int-function-limit}
    \end{equation}
    where $ a(t) $ and $ b(t) $ are differentiable functions.
\end{lemma}

\begin{lemma}\label{lemma:int-function-another-var}
    Let $ f : \mathbb{R} \times \mathbb{R} \rightarrow \mathbb{R} $ be a continuous function such that the partial derivative $ \frac{\partial f}{\partial t} (x, t) $ exists and is continuous in both variables $ x $ and $ t $.
    \begin{equation}
        \frac{d}{dt} \left( \int_{a}^{b} f(x, t) dx \right) = \int_{a}^{b} \frac{\partial f}{\partial t} (x, t) dx.
        \label{eq:lemma:int-function-another-var}
    \end{equation}
\end{lemma}

\begin{lemma}\label{lemma:int-function-general}
    Let $ f(x, t) $ be a continuous function such that the partial derivative $ \frac{\partial f}{\partial t} (x, t) $ exists and is continuous.
    Then,
    \begin{equation*}
        \frac{d}{dt} \left( \int_{a(t)}^{b(t)} f(x, t) dx \right) = \int_{a(t)}^{b(t)} \frac{\partial f}{\partial t} (x, t) + f(b(t), t) b'(t) - f(a(t), t) a'(t).
    \end{equation*}
\end{lemma}
Note that Lemma \ref{lemma:int-function-limit} and Lemma \ref{lemma:int-function-another-var} are special cases of Lemma \ref{lemma:int-function-general}.

\section{Limits}
\begin{definition}
    Let $ g : \mathbb{R} \rightarrow \mathbb{R} $.
    The limit of $ g(x) $ as $ x \rightarrow x_0 $ exists and is finite and equal to $ l $ iff for any $ \epsilon > 0 $ there exists $ \delta > 0 $ such that $ |g(x) - l| < \epsilon $ for all $ x \in (x_0 - \delta, x_0 + \delta) $, i.e.,
    \begin{equation*}
        \lim_{x \rightarrow x_0} g(x) = l, \quad \text{iff} \quad \forall \epsilon > 0 \exists \delta > 0 \quad \text{such that} \quad |g(x) - l| < \epsilon,\ \forall |x - x_0| < \delta.
    \end{equation*}
    Similarly,
    \begin{equation*}
        \lim_{x \rightarrow x_0} g(x) = \infty, \quad \text{iff} \quad \forall C > 0 \exists \delta > 0 \quad \text{such that} \quad g(x) > C,\ \forall |x - x_0| < \delta.
    \end{equation*}
    \begin{equation*}
        \lim_{x \rightarrow x_0} g(x) = -\infty, \quad \text{iff} \quad \forall C < 0 \exists \delta > 0 \quad \text{such that} \quad g(x) < C,\ \forall |x - x_0| < \delta.
    \end{equation*}
\end{definition}

\begin{theorem}
    If $ P(x) $ and $ Q(x) $ are polynomials and $ c > 1 $ is a fixed constant, then
    \begin{align}
        \lim_{x \rightarrow \infty} \frac{P(x)}{c^x} &= 0, \quad \forall c > 1; \\
        \lim_{x \rightarrow \infty} \frac{\ln |Q(x)|}{P(x)} &= 0.
    \end{align}
\end{theorem}

\begin{lemma}
    Let $ c > 0 $ be a positive constant.
    Then,
    \begin{align}
        \lim_{x \rightarrow \infty} x^{\frac{1}{x}} &= 1; \\
        \lim_{x \rightarrow \infty} c^{\frac{1}{x}} &= 1; \\
        \lim_{x \searrow 0} x^x &= 1, \\
    \end{align}
    where the notation $ x \searrow 0 $ means that $ x $ goes to 0 while always being positive, i.e., $ x \rightarrow 0 $ with $ x > 0 $.
\end{lemma}
